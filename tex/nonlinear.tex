\section{Non-linear rheologies}\label{sec:nonlinear}
Non-linear rheologies are implemented combining viscous creep (dislocation and diffusion) and plastic yielding. For each marker, diffusion and dislocation
viscosity ($\eta_{df|ds}$) can be determined as follows \citep{Karato1993,Warren2008b,Wang2016}:
\begin{equation}\label{eq:viscous}
\eta_{df|ds}=f_s \left(\frac{d^m}{A}\right)^{\frac{1}{n}} \left (\frac{2^{\frac{1-n}{n}}}{3^{\frac{n+1}{2n}}} \right)
I_2^{\frac{(1-n)}{n}}\exp\left(\frac{Q+pV}{nRT}\right)
\end{equation}
where $f_s$ is a scaling factor used to represent lithologies that are stronger or weaker than the base set, $d$ is the grain size, $m$ is the grain size exponent,
$n$ is the stress exponent, $A$ is the uniaxial pre-exponential factor, $I_2$ is the square root of the second invariant of the strain rate tensor, $Q$ is the
activation energy, $p$ is the pressure, $V$ is the activation volume, $R$ is the gas constant and $T$ is the temperature.
Pressures and temperatures are determined by the interpolation of the nodal parameters, while strain rates are calculated by means of the
derivative of the velocities on the nodes. In case of diffusion creep $n=1$ and $m>0$, in case of dislocation creep $n>1$ and $m=0$. The viscous creep $\eta_{cp}$
is then calculated as the harmonic average between $\eta_{df}$ and $\eta_{ds}$:
\begin{equation}\label{eq:visco_creep}
\eta_{cp}=\left(\frac{1}{\eta_{df}}+\frac{1}{\eta_{ds}}\right)^{-1}
\end{equation}
The implementation of the non-linear viscous creep viscosity depending on the strain rate is tested by means of the slab detachment benchmark
(Section \ref{sec:slab}).

Plastic yielding is implemented rescaling $\eta_{cp}$ in order to limit the stress below the yield stress $\sigma_y$ \citep{Thieulot2008,Thieulot2014,Glerum2018},
obtaining
\begin{equation}\label{eq:plastic}
\eta_{pl}=\frac{\sigma_y}{2I_2}
\end{equation}
where $\eta_{pl}$ is the plastic viscosity and the yield stress is determined following the Drucker-Prager criterion, such as
\begin{equation}\label{eq:yield}
\sigma_y=C\;\cos(\phi)+p\;\sin(\phi)
\end{equation}
where $C$ is the cohesion and $\phi$ is the internal friction angle. In case of negative pressure it is imposed equal to 0, so that negative yield stress are
excluded. The correctness of the non-linear solution in case of plastic viscosity with variable internal friction angle is verified performing indenter and
brick experiments (Sections \ref{sec:indenter} and \ref{sec:brick}, respectively).

Plastic yielding and viscous creep are then combined to obtain a viscoplastic viscosity $\eta_{vp}$ as follows, assuming that they are independent processes
\citep{Karato2008,Glerum2018}:
\begin{equation}\label{eq:viscoplastic}
\eta_{vp}=\min(\eta_{cp},\eta_{pl})
\end{equation}
Finally, effective viscosity $\eta_{eff}$ is capped by the minimum and the maximum viscosity ($\eta_{min}$ and $\eta_{max}$, respectively) to avoid extremely
low or high viscosity \citep{Glerum2018} as follows
\begin{equation}\label{eq:effective}
\eta_{eff}=\min(\max(\eta_{vp},\eta_{min}),\eta_{max})
\end{equation}
Viscous creep and plastic yielding are non-linear rheologies because of their dependence on the velocity field through pressure and strain rates. Therefore,
the solution is determined by means of Picard-type iterations, until convergence of the velocity field \citep{Glerum2018}. The convergence is verified at each
iteration $i$ via the nonlinear residual $\bm{\mathcal{R}}^i$ that can be determined as
\begin{equation}\label{eq:convergence}
\bm{\mathcal{R}}^i= \mathbb{K} (\eta_{eff}(\dot{\bm{\epsilon}}^{i-1},p^{i-1}))\cdot \bm{v}^{i-1} - \bm{f}^i
\end{equation}
%\begin{equation}\label{eq:convergence}
%\chi=1-\frac{\langle (\bm{v}^i-\langle\bm{v}^i\rangle)\cdot (\bm{v}^{i+1}-\langle\bm{v}^{i+1}\rangle) \rangle}{\sqrt{\sigma^i\sigma^{i+1}}}
%\end{equation}
%Iterations are performed until either convergence ($\chi<tol$) or maximal number of non-linear iterations ($it_{max}$) is reached.
where $\mathbb{K}$ is the Stiffness matrix and $\bm{f}$ is the right hand side vector \citep{Spiegelman2016,Glerum2018}.  The $L_2$-norm is extracted from
$\bm{\mathcal{R}}^i$ and it is normalised as follows
\begin{equation}\label{eq:convergence_res}
\frac{||\bm{\mathcal{R}}^i||_2}{||\bm{\mathcal{R}}^0||_2}
\end{equation}
where $||\bm{\mathcal{R}}^0||_2$ is the $L_2$-norm of the first nonlinear iteration.
Since often the normalised nonlinear residual drop very quickly over the first 2-3 iterations, also the $L_2$-norm of velocity and pressure residuals are
calculated as
\begin{equation}\label{eq:convergence_vel}
\frac{||\Delta\bm{v}||_2}{||\bm{v}||_2}=\frac{||\bm{v}_i-\bm{v}_{i-1}||_2}{||\bm{v}_{i}||_2}
\end{equation}
and
\begin{equation}\label{eq:convergence_press}
  \frac{||\Delta p||_2}{||p||_2}=\frac{||p_i-p_{i-1}||_2}{||p_{i}||_2}
\end{equation}
respectively. Iterations are performed until either maximal number of non-linear iterations ($it_{max}$) is reached or all the normalised nonlinear residual, 
the velocity residual and the pressure residual converge under a defined tolerance $tol$.