\section{Introduction}\label{sec:intro}
This document includes a description of the code FALCON, with a presentation of the features implemented followed by results of extensive benchmarks listed in
Table \ref{tab:list}, performed to verify that the code solves correctly a wide range of problems. Description and results of all benchmarks are shown
in Section \ref{sec:benchmark}.

\begin{table}[H]
\caption{List of benchmarks performed for each feature of the code. Data of all benchmarks can be found at 
\url{https://github.com/aleregorda/Benchmarks}}
\centering
\begin{tabular}{| l | l | l |}
\toprule
 \textbf{Feature}  & \textbf{Benchmark} & \textbf{Section} \\
\midrule
  Solver  & Stokes flow & \ref{sec:stokes} \\
\hline
  \multirow{2}{*}{Markers advection} & Zalesak disk & \ref{sec:runge} \\
    & Conservative Interpolation Velocity & \ref{sec:CVI} \\
\hline
  \multirow{4}{*}{Momentum equation}  & Poiseuille flow & \ref{sec:poiseuille}\\
    & Instantaneous 2D sphere & \ref{sec:ist_sphere} \\
    & Rayleigh-Taylor experiment & \ref{sec:rayleigh} \\
    & Falling block & \ref{sec:block} \\
\hline
    & 2D Stokes sphere & \ref{sec:sphere} \\
    Sticky air and  & Stabilisation algorithm & \ref{sec:stab} \\
    free surface & Topography relaxation & \ref{sec:crameri} \\
  & Spontaneous subduction & \ref{sec:subduction} \\
\hline
  Erosion and sedimentation  & - & - \\
\hline
   Non-linear  & Slab detachment & \ref{sec:slab} \\
   visco-plasticity & Indenter & \ref{sec:indenter} \\
    & Brick & \ref{sec:brick} \\
\hline
  \multirow{3}{*}{Energy equation}  & Advection stabilisation & \ref{sec:advection} \\
    & Simple shear heating & \ref{sec:simple_shear} \\
    & Shear and adiabatic heating & \ref{sec:shear} \\
\hline
\multirow{3}{*}{Energy + momentum} & Mantle convection & \ref{sec:mantle} \\
& Viscoplastic mantle convection & \ref{sec:visco_mantle} \\
& Thin layer entrainment & \ref{sec:thin} \\
\hline
  Phase changes  & \multirow{2}{*}{Hydrated sinking cylinder} & \multirow{2}{*}{\ref{sec:quinquis}} \\
  and hydration  &  & \\
\hline
  Melting & Experimental melting curves & \ref{sec:katz} \\
\bottomrule
\end{tabular}
\label{tab:list}
\end{table}

The code is written in Fortran90 and uses the finite element method (FEM) with quadrilateral $Q_1 \times P_0$ elements (continuous bilinear velocity and
discontinuous constant pressure), associated with MUMPS\footnote{\url{http://mumps.enseeiht.fr/}} \citep{Amestoy2001,Amestoy2019} that is a software package
for solving systems of linear equations of the form $A \cdot x = b$, where \textit{A} is a square sparse matrix, by means of a direct method. Since
$Q_1 \times P_0$ elements do not satisfy Ladyzhenskaya, Babuska and Brezzi (LBB) stability condition \citep{Donea2003}, elemental pressure is elaborated
in post-processing to avoid spurious pressures by means of a double interpolation to smooth the pressure. Following this procedure, the elemental pressure
is interpolated onto nodes and then back onto elements. After the smoothing procedure, the pressure field is calculated again on the nodes to be used in
combination with the lithostatic pressure, which is calculated onto nodes. Correctness of the solution and performances of the solver in terms of time and
memory usage are tested solving the Stokes flow with the analytical solution proposed by \citet{Donea2003} (Section \ref{sec:stokes}).

The thermo-mechanics of crust-mantle systems is described by means of the conservation of mass, momentum and energy equations, expressed as follows:
\begin{eqnarray}
\label{eq:mass}\frac{\partial \rho}{\partial t}+\bm{\nabla} \cdot (\rho \bm{u})&=&0\\
\label{eq:momentum}-\bm{\nabla} p + \bm{\nabla} \cdot \bm{\tau} + \rho \bm{g}&=&0\\
\label{eq:energy}\rho C_p \left(\frac{\partial T}{\partial t} + \bm{u} \cdot \bm{\nabla} T \right)&=&\bm{\nabla} \cdot \left(k\bm{\nabla} T\right) + H_{tot}
\end{eqnarray}
where $\rho$ is the density, $\bm{u}$ is the velocity, \textit{p} is the pressure, $\bm{\tau}$ is the deviatoric stress, $\bm{g}$ is the gravity acceleration,
$C_p$ is the specific heat at constant pressure, \textit{T} is the temperature, \textit{k} is the thermal conductivity and $H_{tot}$ is the total internal heat
production. Density variations due to temperature are generally small enough to assume the density as constant ($\rho=\rho_0$) in Eq. \ref{eq:mass} and in
Eq. \ref{eq:energy}, while it must be treated as a variable in the buoyancy term of Eq. \ref{eq:momentum}, such that \[\rho=\rho_0(1-\alpha(T-T_0))\]
where $\rho_0$ is the density at a reference temperature $T_0$ and $\alpha$ is the coefficient of thermal expansion. Eq. \ref{eq:mass} then can be rewritten as
\begin{equation}\label{eq:mass_0}
\bm{\nabla} \cdot \bm{u}=0
\end{equation}
This simplification is known as the Boussinesq approximation.

The time step $dt$ is chosen in according to Courant-Friedrichs-Lewy condition \citep{Anderson1995}, such that
\[dt=C_n\; \min\left(\frac{\Delta x}{\max|\bm{u}|},\frac{(\Delta x)^2}{\kappa}\right)\]
where $0<C_n<1$ is the Courant (CFL) number, $\Delta x$ is the minimum dimension of smallest element, $\max|\bm{u}|$ is the maximum velocity calculated on the
entire domain and
$\kappa=\frac{k}{\rho C_p}$ is the thermal diffusivity \citep{Thieulot2014}.