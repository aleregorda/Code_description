\section{Phase transitions and hydration}\label{sec:phase}
Variations of effective density, specific heat and coefficient of thermal expansion during the evolution of crustal and mantle materials at different pressure-temperature ($p$-$T$) conditions are computed by means of version 6.8.6 of Perple\_X software package \citep{Connolly2005}, similarly to the implementation described in \citet{Marotta2020}. By default, mineral assemblages and properties of each lithology (oceanic crust, lower and upper continental crust, sediments and dry or hydrated mantle) are calculated for temperatures from \SIrange{330}{1600}{\kelvin}, with increments of approximately \SI{4}{\kelvin}, and pressures from \SIrange{0.1}{30}{\giga\pascal}, with increments of approximately \SI{0.03}{\giga\pascal}, for a total of almost 100000 points.

The introduction of phase transitions produces two main effects that must be taken into account in numerical models: variations in density and release/absorption of latent heat that is required by the extended Boussinesq approximation \citep{Tackley2010}. Effects of density variations on buoyancy force in the momentum equation (Eq. \ref{eq:momentum_penalty}) are taken into account considering effective coefficient of thermal expansion \citep{Christensen1985,Zhong2015}. Similarly, effects of latent heat in the energy equation (eq \ref{eq:energy}) are taken into account by considering effective specific heat and coefficient of thermal expansion
\citep{Christensen1985,Tackley2010,Zhong2015}. Effective density, specific heat and coefficient of thermal expansion are included in Perple\_X files and are assumed by each Lagrangian marker in according to its $p$-$T$ conditions.

Hydration processes related to dehydration of subducting lithosphere \citep{Schmidt1998,Liu2007,Faccenda2009,Faccenda2010,Faccenda2014,Rosas2016} strongly
influence the thermo-mechanics inside the mantle wedge, mainly because of both weakening effects on the mantle rheology and density variations in case of mantle serpentinization \citep{Gerya2002,Honda2003,Arcay2005,Roda2010,Regorda2017}. In the case that hydration is switched on, the amount of bound and free water is memorised by each marker, following the implementation of \citet{Quinquis2014}; hydration and dehydration processes are related to the amount of bound water of each marker and to the maximum amount of water it can transport, i.e. if the amount of bound water exceeds the maximum amount of water, the marker dehydrates and releases free water that can hydrates under-saturated markers. Maximum water content of each marker is determined as function of lithology and $p$-$T$ conditions and it is calculated using Perple\_X, in the same way than effective density, specific heat and coefficient of thermal expansion. Bound water is advected together with the markers, neglecting the effect of bound water diffusion, while free water simply migrates vertically and is not coupled to the solid-phase flow of the mantle wedge \citep{Arcay2005,Quinquis2014}. The correctness of the migration scheme of the free water and its absorption into bound
water by under-saturated markers are verified performing the experiment described in \citet{Quinquis2014} (Sec. \ref{sec:quinquis}).

The mantle viscosity weakening related to the amount of water has been extensively studied \citep{Chopra1981,Hirth2003} and it is implemented in according to \citet{Arcay2005} and \citet{Horiuchi2016}, as follows
\begin{equation}\label{eq:wet_visc}
\eta_{wet}=\eta_{dry}\left(\left[1-\frac{1}{f_v}\right]\exp\left(-\frac{[OH^-]}{[OH^-]_0}\right)+\frac{1}{f_v}\right)
\end{equation}
where $f_v$ is the viscosity reduction factor between dry and wet conditions, $[OH^-]$ is the water content and $[OH^-]_0$ is a reference water content set to 620 ppm (0.062 wt.\%) \citep{Arcay2005}. By default, $f_v$ is set to 100, in according to \citet{Arcay2005} and \citet{Horiuchi2016}.