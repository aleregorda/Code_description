\section{Sticky air and free surface}\label{sec:sticky}
The Earth's surface can be treated by means of either the so-called sticky-air or a true free surface method, both of them implemented in the code. In the
sticky-air method the surface is approximated with the introduction of a buoyant layer with a viscosity at least four orders of magnitude lower than the crust
\citep{Schmeling2008,Crameri2012} and the interface between lithosphere and air is defined using a chain of passive markers that are advected as the Lagrangian
markers. The correctness of the evolution of the markers chain is tested with the experiment of a 2D time-dependent Stokes sphere below a free surface and
compared with results from ASPECT\footnote{\url{https://aspect.geodynamics.org/}} \citep{Kronbichler2012,Heister2017,Bangerth2020,Bangerth2020a} (Section
\ref{sec:sphere}). In the true free surface case the top boundary is assumed stress-free and velocities are not fixed. In this case topography variations are
described by vertical deformations of the mesh that depend on the velocity field of the nodes that identify the free surface, while horizontal deformations are
not taken into account. This procedure is known as the Arbitrary Lagrangian-Eulerian (ALE) method and its implementation follows the technique described in
\citet{Thieulot2011}. However, although the implementation of a true free surface better reproduces laboratory experiment, extremely small time steps can be
necessary to maintain stability \citep{Kaus2010a,Quinquis2011,Thieulot2014}. Therefore, the stability algorithm proposed by \citet{Kaus2010a} is implemented
to avoid instabilities due to high density differences at the free surface when using too large time steps. The implementation of this algorithm are tested by
performing the experiment described by \citet{Kaus2010a} (Section \ref{sec:stab}).

The topography relaxation benchmark proposed by \citet{Crameri2012} is performed to verify that the code correctly recovers topography variations in case of
both the sticky-air and the true free surface method (Section \ref{sec:crameri}). Finally, Section \ref{sec:subduction} show the results of the spontaneous
subduction experiment described by \citet{Schmeling2008}.