\section{Lagrangian markers}\label{sec:markers}
Elemental properties (density, viscosity, thermal conductivity, specific heat and thermal expansion) needed to solve Eqs. \ref{eq:mass_0}, \ref{eq:momentum}
and \ref{eq:energy} are related to the composition of each element, determined by means of Lagrangian markers that are characteristic of different materials
in the domain. Elemental properties (with the sole exception of the viscosity) are calculated using an arithmetic mean, as
\begin{eqnarray}
P_e=\frac{1}{n}\sum_{i=1}^n P_i
\end{eqnarray}
where $P_e$ is the elemental property, $P_i$ is the property characteristic of the material of each marker and $n$ is the number of markers of the element.
Differently, the average scheme for the elemental viscosity can be chosen between harmonic, geometric and arithmetic mean. Markers can be placed regularly or
randomly at the beginning of the simulation and their advection is performed by either a 2nd-order or a 4th-order Runge-Kutta in space, interpolating the velocity 
field on each marker by means of the shape functions (see Section \ref{sec:runge}). The interpolated velocity is then corrected by means of the Conservative
Velocity Interpolation (CVI), which introduces a corrective term to reduce dispersion and clustering of particles in both steady state and time-dependent 
\citep{Wang2015} (see Section \ref{sec:CVI}).

The initial distribution of the markers is created using the open-source code library
Geodynamic World Builder \citep{Fraters2019}. During the simulation, each marker carries memory of temperature, pressure and accumulated strain, which are
determined interpolating the relative nodal parameter. In particular, pressure is interpolated onto each marker after the smoothing procedure. The number
of markers contained in each element is maintained between $n_{min}$ and $n_{max}$. When in an element there are less markers than $n_{min}$ the code adds
random markers to reach the $n_{min}$, while if the number is higher than $n_{max}$ some of them are randomly deleted. When new markers are added, they assume
the properties of the nearest marker. In this way elements are never empty and maintain a number of markers inside a prefixed range.